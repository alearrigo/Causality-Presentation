\documentclass[12pt]{exam}
\usepackage[utf8]{inputenc}

\usepackage[margin=1in]{geometry}
\usepackage{amsmath,amssymb}
\usepackage{multicol}

\newcommand{\class}{Metodi statistici non parametrici}
\newcommand{\term}{28 Aprile 2017}
%\newcommand{\examnum}{Prova in itinere}
%\newcommand{\examdate}{5/5/2016}
\newcommand{\timelimit}{30 Minuti}

\pagestyle{head}
\firstpageheader{}{}{}
\runningheader{\class}{\examnum\ - Page \thepage\ of \numpages}{\examdate}
\runningheadrule


\begin{document}

\noindent
\begin{tabular*}{\textwidth}{l @{\extracolsep{\fill}} r @{\extracolsep{6pt}} l}
\textbf{\class} & \textbf{ \hspace{0.8cm} Nome Cap.: Chiara Di Maria} &\\
\textbf{\term} &&\\
%\textbf{\examnum} &&\\
%\textbf{\examdate} &&\\
\textbf{Tempo: \timelimit} & &\\
\textbf{Componenti del gruppo:} & Di Maria - Martello - Rubino - Arrigo&\\
\textbf{} & &\\
\textbf{Argomento:} &  Graphical models for Probabilistic and Causal Reasoning& \\
\textbf{Valutazione:} &   A(28-30)-B(24-27)-C(20-23)-D(18-19)-E($<$18)&\\


%\makebox[2in]{\hrulefill}
\end{tabular*}\\
\rule[2ex]{\textwidth}{2pt}

\noindent
Il capitano di ogni gruppo dovrà esporre l’argomento teorico assegnato per un tempo non superiore ai 40 minuti. I primi 30 minuti saranno dedicati alla presentazione dell'argomento, mentre i restanti 10 minuti saranno utilizzati per rispondere ad alcune domande poste dal docente e dai colleghi. 
Ai gruppi sarà chiesto di preparare, oltre alla presentazione, due domande, da porre ai colleghi. Le slide della presentazione dovranno pertanto essere inviate al docente e agli altri gruppi almeno due giorni prima dalla data prevista per la presentazione. 

\noindent
\rule[2ex]{\textwidth}{2pt}

\noindent
La prova consiste nell'approfondimento teorico dell'articolo \textit{"Graphical Models for Probabilistic and Causal Reasoning"}. In particolare, si deve sviluppare una discussione sui modelli grafici direzionali ("Bayesian Networks") e sulla differenza tra azione (uso dell'operatore "$do$") e osservazione.

Di seguito sono fornite alcune linee guida per affrontare la prova. 
\begin{itemize}
\item Introduzione ai modelli grafici direzionali (Bayesian Networks). 
\item Ragionamento casuale e uso dell'operatore "$do$".
\item Legame con i modelli ad equazioni strutturali. 
\end{itemize}

\noindent
La struttura della presentazione potrebbe essere suddivisa in quattro parti.

\begin{itemize}
\item Una breve introduzione ai modelli grafici direzionali.
\item Ragionamento casuale e modelli ad equazioni strutturali.
\item Un'applicazione su dati reali.
\item Conclusioni.
\end{itemize}

\noindent
\rule[2ex]{\textwidth}{2pt}
\noindent
\textbf{Bibliografia}\\
\noindent
Judea Pearl. \textit{Graphical Models for Probabilistic and Causal Reasoning}. Computing Handbook, Third Edition: Computer Science and Software Engineering, Volume I (2013).

\end{document}
